\documentclass[acmlarge]{acmart}

\usepackage{booktabs} % For formal tables


\usepackage[ruled]{algorithm2e} % For algorithms
\renewcommand{\algorithmcfname}{ALGORITHM}
\SetAlFnt{\small}
\SetAlCapFnt{\small}
\SetAlCapNameFnt{\small}
\SetAlCapHSkip{0pt}
\IncMargin{-\parindent}
\newcommand{\Mod}[1]{\ (\mathrm{mod}\ #1)}
% Metadata Information
%\acmJournal{PACMHCI}
%\acmVolume{9}
%\acmNumber{4}
%\acmArticle{39}
%\acmYear{2010}
%\acmMonth{3}
%\acmArticleSeq{11}


% DOI
\acmDOI{0000001.0000001}

% Paper history
%\received{February 2007}
%\received{March 2009}
%\received[accepted]{June 2009}


% Document starts
\begin{document}
% Title portion
\title{Security Suite: Team 4 EECS 444 Final Project}
% \titlenote{We can add a note to the title}

\author{Kim Almcrantz}
%\affiliation{\institution{Case Western Reserve University}}
\email{kaa97@case.edu}

\author{Mark Lalor}
%\affiliation{\institution{Case Western Reserve University}}
\email{mwl58@case.edu}

\author{Brian Li}
%\affiliation{\institution{Case Western Reserve University}}
\email{bvl8@case.edu}

\author{Vanessa Melikian}
%\affiliation{\institution{Case Western Reserve University}}
\email{vlm21@case.edu}

\author{Maya Nayak}
%\affiliation{\institution{Case Western Reserve University}}
\email{mkn30@case.edu}

\author{Jacob Wise}
%\affiliation{\institution{Case Western Reserve University}}
\email{jsw107@case.edu}

\keywords{encryption, cipher, hash, entropy}

\maketitle

% The default list of authors is too long for headers.
% \renewcommand{\shortauthors}{G. Zhou et al.}

\section{Abstract}
Sensitive data, including things such as bank details, and personal information, must be continuously transferred amongst machines in order for society to function. In an attempt to fend off malicious users attempting to gain access to this data, digital information must be hidden. A common way to do this is through some type of encryption. Cryptography is a large field, with many different techniques. Each technique operates best given certain assumptions. In this paper, we aim to explain and implement the following cryptographic techniques: RSA, DES, and Vigenere Cipher. Each works best under different conditions; for instance, DES operates at the bit level, while Vigenere is at the character level. Through our tool, we can see at first glance that the information seems hidden. But, we will also explore the encryption cracking of Vigenere enciphered and MD-5 enciphered text. This illustrates that many cryptographic techniques have vulnerabilities that can be exploited by ill-intentioned users. 


\section{Introduction}\label{sec:intro}

Modern communications require special methods to ensure the security of information in the presence of adversaries that want to intercept or maliciously modify communicated data. An adversary may try to compromise the confidentiality, or integrity of data, or they may bypass an authentication completely with a carefully-crafted message.

This article is divided into several sections, in Section [\ref{sec:intro}], we introduce the cryptographic environment that we will explore. In Section [\ref{sec:algorithms}] we describe implementations of several symmetric and asymmetric encryption algorithms, and then Section [\ref{sec:gui}] we present our tool \textsc{SecuritySuiteGUI}, a software suite to experiment with these implementations. We then demonstrate the power of password cracking with a hashcat demo in Section [\ref{sec:hashcat}]. Finally, we demonstrate how the Vigenere cipher may be easily cracked with computational power in Section [\ref{sec:vinegar}]. We evaluate our methods in Section [\ref{sec:evaluation}], and then discuss our techniques, challenges, and other thoughts in Section [\ref{sec:discussion}]. Finally, we discuss our final conclusions in Section [\ref{sec:conclusions}].

\subsection{Overview of Cryptography Algorithms}
\subsubsection{DES}
\hspace*{\fill} \\ % force newline
Data Encryption Standard (DES) is another type of data encryption described as a symmetric-key block cipher. A symmetric key uses the same key to encrypt and decrypt a message, which means that both the sending and receiving side of the transaction must use the same private key. A block cipher takes in a string of bits and transforms it through bit-wise operations into a new string of bits with the same length. 

DES begins by taking in 2 64-bit inputs: a key and message. In general, all operations are done on the bit-level, using tables of predetermined size and content. The algorithm then permutes the input key according to a given table, PC-1. This results in a new 56-bit key. The new key is then split into two 28-bit halves, $C_{0}$ and $D_{0}$. Using these halves, we iterate 16 times to create 16 sets of blocks (n from 1 to 16). Every $C_{n}$ and  $D_{n}$ is computed by performing a set number of left shifts (either one or two) on $C_{n-1}$ and $D_{n-1}$.
 The resulting pairs are concatenated into a 56-bit $C_{n}D_{n}$, and permuted with the table PC-2 to form 16 keys, $K_{n}$. At this point, we have the 16 subkeys (of size 48 bits) necessary. 

Moving to the encryption of the message itself, we begin by permuting it with the table IP. The resulting permutation is then split into 32-bit halves, $L_{0}$ and $R_{0}$. Over the course of 16 iterations, we create 16  sets of blocks using the Feistel function f:
$L_{n} = R_{n-1}$
$R_{n} = L_{n-1} \bigoplus f(R_{n-1}, K_{n})$ 
	The Feistel function involves taking in a subkey $K_{n}$ and block $R_{n - 1}$. It permutes $R_{n - 1}$ using table E, and then XORs the result with $K_{n}$. It then takes the resulting block and permutes it using a series of 8 S-boxes. There is then a final permutation on the S-box output with table P. At the end of the iterations we are left with a concatenated block, $R_{16}L_{16}$. This block is finally permuted with a table $IP^{-1}$, and returned as the output of the DES algorithm. Note that encryption and decryption are almost exactly the same, except for the order in which the subkeys are applied. For encryption, the keys are applied $K_{1}$ to $K_{16}$ and for decryption they are applied $K_{16}$ to $K_{1}$.

While DES used to be heavily adopted, due to the small size of the key it has been found to be insecure by a brute force attack. It is no longer used as a standard. Though Triple DES, an extension of DES, is thought to be secure, its implementation is out of the scope of this paper. 


\section{Tool Design} \label{sec:impl}
\subsection{Algorithm Implementations} \label{sec:algorithms}
\subsubsection{DES}
\hspace*{\fill} \\ % force newline
In order to produce the implementation of DES, we modularized the functionality of each component in DES while also allowing for our program to interface with a GUI. Each of the high level functions are explained below. 

Interfacing - Encipher / Decipher: These methods were used to interface with a GUI. Per our teams restrictions, we input a key and message in the form of a byte array. Since it was possible for the key and message to not be 64 bytes, we implemented a set of conditionals to pad our key or message. Then, we converted the byte arrays (GUI inputs) into integer values that could be used as input to our encryption function. We saved the result and converted it to a byte array as output to be displayed in the GUI. It is important to note that only one method is ideally needed for encipher and decipher, but because the interface implementation included a separate encipher and decipher, our code had to mimic the same structure. 

Permute Using Table: This method was used to implement a permutation on a table. Multiple permutations are undergone throughout DES; this function allows a user to specify on which table one is permuting given the table, block, and its length, while returning the result of that permutation. 

Make Round Keys: Since DES requires 16 keys, this function takes each 28 bit half-key and first performs a left rotating function (a helper method we implemented to aid in the left shift of either 1 or 2 bits). We do this 16 times to produce 16 pairs of half-subkeys. These subkeys are populated into an array, in which each half is concatenated together. Our "permute using table" function is called to permute our 56 bit keys into table PC2, from which we return 16 48-bit subkeys.

Round Function: The last job of DES encryption is to perform a feistel function. Our round function will perform one set of feistel operations given a singular key and block. The given 32-bit block is first permuted using an expansion table to yield a 48-bit block. We then XOR this 48-bit block with the 48-bit key to produce a new block, which is then split into 8 6-bit blocks. Using a loop, we enumerate over these blocks in order to permute them into an S-box (where each 6-bit block is converted to a 4-bit block). A final permutation is performed with a P-box using "permute using table". 

Encryption: This method encapsulates all methods described above (outside of the interfacing methods). Encryption is called within the interfacing methods encipher/decipher. The encryption method first checks that the type of the key and message are integers and that they are at most 64 bits in length. We then permute the key using table PC1 and the key is split into two 28 bit halves. Then, "Make round key" is called to make our 16 subkeys (as described above), which yields 48-bit keys. Our block is then permuted using table IP. Our block message is then split into two 32 bit halves, after which we enumerate over all 16 subkeys and perform the feistel function. The cipher-blocks are then rejoined and permuted on table inverse IP. This results in our final ciphertext. 

\subsubsection{RSA}
% Algorithm
\begin{algorithm}[tbh]
\label{rsa_algo}
\SetAlgoNoLine
\KwIn{Public key exponent $e$, key size $s$}
\KwOut{Public key $k_{pub}$, and private key $k_{priv}$}

\Repeat{$\gcd(e, \lambda) = 1 \land \|p - q\| \geq 2^{s / 2 - 100}$}{
    $p \longleftarrow \text{probable-prime}(s / 2)$

    $q \longleftarrow \text{probable-prime}(s / 2)$

    $\lambda \longleftarrow \text{lcm}(p - 1, q - 1)$
}

$k_{pub} \longleftarrow pq$

$k_{priv} \longleftarrow e^{-1} \mod{\lambda}$

\KwRet $k_{pub}$, $k_{priv}$
\caption{RSA Keygen implementation}
\label{alg:rsa-keygen}
\end{algorithm}

Unlike DES and Vignere, RSA is an asymmetric cipher. Specifically, this means that the RSA cipher (i.e. public key) used to encrypt a message cannot be used to decrypt the the encrypted message whereas the cipher used to encrypt a message in DES and Vignere can be used to decrypt the encrypted message by reversing the encryption process.
		
To use RSA, a user first generates a key pair. The key pair contains a private key, public key, and a public key exponent. The public key and public key exponent can be made public so that other users can send the original user an encrypted message. The original user can use his/her private key to decrypt messages encrypted with the published public key and exponent. The main benefit that asymmetric encryption provides is that communicating entities can securely communicate without exchanging a decryption key/cipher.

The key mechanic in RSA comes from the following congruence:

\begin{equation}
\label{rsa_mechanic}
	m^{ed} \equiv m \Mod{pq}
\end{equation}

When ed satisfies:

\begin{equation}
\label{ed_equiv}
	ed \equiv 1 \Mod{\lambda(pq)}
\end{equation}
\begin{equation}
\label{carmichael}
	\lambda(pq) = lcm(p - 1, q - 1)
\end{equation}

To see why \ref{rsa_mechanic} is true, we will give a proof:
\begin{equation}
\label{rsa_proof}
\begin{split}
	m^{ed} 
	\equiv m^{k\lambda(pq) + 1} \Mod{pq} \\
	\equiv m * m^{k\lambda(pq)} \Mod{pq} \\
	\equiv m * (m^{\lambda(pq)})^{k} \Mod{pq} \\
	\equiv m * (1)^{k} \Mod{pq} \\
	\equiv m
\end{split}
\end{equation}

To get:
\begin{equation}
	m * (m^{\lambda(pq)})^{k} \Mod{pq} \equiv m * (1)^{k} \Mod{pq}
\end{equation}

We use the Carmichael function ($\lambda$) which gives us the congruence:
\begin{equation}
\label{carmichael_fun}
	a^{\lambda(n)} \equiv 1 \Mod{n}
\end{equation}

When a and n are co-prime i.e. gcd(a, n) = 1

Using \ref{rsa_mechanic}, we can let e, d, and n be the public key exponent, private key, and public key respectively. If m is the message, we can encrypt messages with the public key and public key exponent by:

\begin{equation}
	m^{e} \equiv c \Mod{n}
\end{equation}

Where c is the encrypted message. To reverse the encryption, one would need to take the $e^{th}$ root of c (mod n) which is computationally difficult (citation needed). However, with the private key and our proof in \ref{rsa_proof}, we can decrypt c:

\begin{equation}
	c^{d} \equiv (m^{e})^{d} \equiv m^{ed} \equiv m \Mod{n}
\end{equation}

Now that we have shown the correctness and mechanic of RSA, we will now discuss the algorithm to generate a key pair consisting of a public key exponent, public key, and private key. Algorithm \ref{rsa_algo} gives pseudocode for generating an RSA key pair. In the algorithm, two large, random numbers are generated. Since it is computationally expensive and impractical to check if the generated numbers are prime with certainty, Miller-Rabin is used to probabilistically check the primality of the generated numbers (citation needed). The more rounds of Miller-Rabin a number passes, the more likely the number is prime (citation needed). In Java's BigInteger class, the probablePrime() function returns a number with probability $1 - 2^-100$ that the generated number is prime (citation needed). After generating two probable prime numbers, p and q, we make two checks. First, we check that e and $\lambda$ are co-prime so that we can use the Carmichael function congruence shown in \ref{carmichael_fun}. Second, we check that p and q are at least a few magnitudes apart to make factorizing pq more difficult (citation needed). Finally, we return pq as the public key, e as the public key exponent, and $e^{-1} \Mod{pq}$ as the private key.

%RSA Citations:
%https://docs.oracle.com/javase/7/docs/api/java/math/BigInteger.html#probablePrime(int,%20java.util.Random)
%http://hg.openjdk.java.net/jdk8/jdk8/jdk/file/tip/src/share/classes/java/math/BigInteger.java
%Rivest, R.; Shamir, A.; Adleman, L. (February 1978). "A Method for Obtaining Digital Signatures and Public-Key Cryptosystems" (PDF). Communications of the ACM. 21 (2): 120–126. doi:10.1145/359340.359342. 
% The above citation is used for why the primes need to be a certain distance apart.

\subsubsection{Vigenere Cipher}
\hspace*{\fill} \\ % force newline
\textit{TODO: Vigenere Cipher Description}

\subsection{SecuritySuiteGUI}\label{sec:gui}

% Figure
\begin{figure}
  \centering
  \includegraphics[scale=0.40]{demo}
  \Description{Image of our demo tool}
  \caption{Image of the security suite demo tool.}
  \label{fig:one}
\end{figure}

\subsection{Hashcat Demo}\label{sec:hashcat}

\textsc{Hashcat} is a software tool used to crack passwords by leveraging the highly-parallelized general purpose computation capabilities of GPU \cite{Hashcat}. \textsc{Hashcat} will calculate hashes based on one of several attack modes, and the hashes are automatically checked against the hashes selected to be attacked.

The most important option when using \textsc{Hashcat} is the \texttt{hash-type}, which specifies the actual hash algorithm that will be used for the attack. For example, you can choose a simple hash algorithm such as \texttt{"md5"}, or you may choose a composition of algorithms such as \texttt{"sha1(\$salt.sha1(\$pass))"}.

Depending on the algorithm, \textsc{Hashcat} may or may not be able to leverage the GPU's capabilities effectively. Some algorithms, such as \texttt{bcrypt} are specifically designed to be difficult to parallelize in GPU hardware. Despite having thousands of cores to parallel computation, the algorithm is bottlenecked by the shared memory bus since the algorithm requires operating on a shared block of memory. Other algorithms, such as \texttt{MD5} are able to be run on the order of \textit{trillions} of time per second on modern hardware setups.

The second required option is an \texttt{attack mode}. The simplest is a \texttt{brute-force attack}, which tries all possible combinations in a given keyspace. For example, a brute-force attack may be done on the following charset:

\begin{center}
\texttt{abcdefghijklmnopqrstuvwxyz0123456789}	
\end{center}

If we use this in conjunction with a password length of $6$, this will try all passwords composed of $6$ characters from that charset. For example, the password \texttt{k39a1j} will be discovered this way.

When trying to crack many passwords, the simple \texttt{brute-force attack} is usually an unrealistic attack method. The \texttt{mask attack} can use multiple charsets and string patterns to reduce the password candidate keyspace. 

\section{Vigenere Cipher Cracker}\label{sec:vinegar}

Words go here too!

\section{Results}\label{sec:results}

Evaluation, efficiency? Challenges?

\section{Related Work}\label{sec:relatedwork}

Related work. Examples of citations with DOIs: \cite{2004:ITE:1009386.1010128, Kirschmer:2010:AEI:1958016.1958018}. Online citations: \cite{TUGInstmem, Thornburg01, CTANacmart}.

\section{Discussion}\label{sec:discussion}

What didn't we cover? :O

% Appendix
\appendix
\section{Elaboration on the ABCD algorithm}

This is an appendix, maybe about some equation
\begin{displaymath}
P=NP
\end{displaymath}

\section{Supplementary Materials}

\subsection{Hashcat materials}

Materials?

\subsection{Tool: Symmetric Ciphers Online }

\href{http://symmetric-ciphers.online-domain-tools.com/}{Link}

\begin{acks}

The authors would like to thank Professor Xiao and the Case Western Reserve EECS Department.

\end{acks}

% Bibliography
\bibliographystyle{ACM-Reference-Format}
\bibliography{bibliography}

\end{document}